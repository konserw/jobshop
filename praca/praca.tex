%Praca inżynierska (c) Kamil Strzempowicz
\documentclass[twoside]{pracaInzynierskaMS}
%\documentclass[11pt,a4paper]{article}
\usepackage{polski}
%\usepackage[polish]{babel}
\usepackage[utf8]{inputenc}
\usepackage{mathtools}
\usepackage{color}
\usepackage{graphicx}
\usepackage{transparent}
\usepackage{lscape}
\mathtoolsset{showonlyrefs}


%%%%%%%%%%%%%%%%%%%%%%%%%%
%%% Przełączniki stylu %%%
%%%%%%%%%%%%%%%%%%%%%%%%%%

\sectionWzory  
%\drukJednostronny
\literowaNumeracjaDodatkow %% w³¹czy numeracjê dodatków literami
\bezWyjasnien

%%%%%%%%%%%%%%%%%%%%%%%
%%% Podstawowe info %%%
%%%%%%%%%%%%%%%%%%%%%%%

\title{Strategia Just in Time w systemach produkcyjnych - analiza struktury gniazdowej dla heurystyk\\ FIFO i LIFO.}
\promotor{dr inż. Waldemar Grzechca}
\autor{Kamil Strzempowicz}{100}{Napisanie całej pracy}
%% dedykacja mile widziana
\dedykacja{Rodzicom\\bez Was nie udałoby mi się}

%%%%%%%%%%%%%%%%%%%%%%%%%%%%%%%%%
%%% początek właściwej treści %%%
%%%%%%%%%%%%%%%%%%%%%%%%%%%%%%%%%

\begin{document}
\section        {Wstęp}

Tutaj będzie wstęp \cite{Poz1} jak tylko go wymyślę, na razie walnę lorem impsum :D \\

\section        [Metody szeregowania zadań \ldots]
		        {Metody szeregowania zadań w systemach wytwarzania gniazdowego}

\subsection     {Metody heurystyczne}
\subsubsection  {First In First Out}

\subsubsection  {Last In First Out}

\subsubsection  {Earliest Due Date}



\subsection     {Inne metody}
\subsubsection  {Algorytmy genetyczne}

\subsubsection  {Metoda mrówkek}


%\newlineTekst
%\newlineSpis
\section        [Heurystyki LIFO i FIFO \ldots]
                {Heurystyki LIFO i FIFO dla gniazdowej struktury systemu produkcyjnego - przykłady obliczeniowy}

\subsection     {Różne rozwiązania}
\subsubsection  {FIFO}

\subsubsection  {LIFO}

\subsection     {Identyczne rozwiązania}
\subsubsection  {FIFO}

\subsubsection  {LIFO}

       
\section        {Program kSzereg}
\subsection     {Front-end}
Program kSzereg został wykonany na potrzeby tej pracy dyplomowej.\\
Cały kod źródłowy znajduje się na załączonej płycie cd-rom. \\
Natomiast istotne metody  zosstały przytoczone w dodtku Program

\subsection     {Back-end}
Jakiś podrożdział żeby było spoko
       
\section        [Przykładowe problemy \ldots]
                {Przykładowe problemy rozwiązane przy pomocy programu kSzereg}
       
\subsection     {Probelm 1}
\documentclass[11pt,a4paper]{article}
\usepackage{polski}
\usepackage[utf8]{inputenc}
\usepackage{mathtools}
\usepackage{color}
\usepackage{graphicx}
\usepackage{transparent}
\mathtoolsset{showonlyrefs}
\title{Generated by kSzereg}
\date{}
\begin{document}
	\documentclass[11pt,a4paper]{article}
\usepackage{polski}
\usepackage[utf8]{inputenc}
\usepackage{mathtools}
\usepackage{color}
\usepackage{graphicx}
\usepackage{transparent}
\mathtoolsset{showonlyrefs}
\title{Generated by kSzereg}
\date{}
\begin{document}
	\documentclass[11pt,a4paper]{article}
\usepackage{polski}
\usepackage[utf8]{inputenc}
\usepackage{mathtools}
\usepackage{color}
\usepackage{graphicx}
\usepackage{transparent}
\mathtoolsset{showonlyrefs}
\title{Generated by kSzereg}
\date{}
\begin{document}
	\input{output/zad1.tex}
\end{document}

\end{document}

\end{document}


\newpage
\subsection     {Problem 2}
\documentclass[11pt,a4paper]{article}
\usepackage{polski}
\usepackage[utf8]{inputenc}
\usepackage{mathtools}
\usepackage{color}
\usepackage{graphicx}
\usepackage{transparent}
\mathtoolsset{showonlyrefs}
\title{Generated by kSzereg}
\date{}
\begin{document}
	\documentclass[11pt,a4paper]{article}
\usepackage{polski}
\usepackage[utf8]{inputenc}
\usepackage{mathtools}
\usepackage{color}
\usepackage{graphicx}
\usepackage{transparent}
\mathtoolsset{showonlyrefs}
\title{Generated by kSzereg}
\date{}
\begin{document}
	\documentclass[11pt,a4paper]{article}
\usepackage{polski}
\usepackage[utf8]{inputenc}
\usepackage{mathtools}
\usepackage{color}
\usepackage{graphicx}
\usepackage{transparent}
\mathtoolsset{showonlyrefs}
\title{Generated by kSzereg}
\date{}
\begin{document}
	\input{output/zad2.tex}
\end{document}

\end{document}

\end{document}


\newpage
\subsection     {Problem 3}
\documentclass[11pt,a4paper]{article}
\usepackage{polski}
\usepackage[utf8]{inputenc}
\usepackage{mathtools}
\usepackage{color}
\usepackage{graphicx}
\usepackage{transparent}
\mathtoolsset{showonlyrefs}
\title{Generated by kSzereg}
\date{}
\begin{document}
	\documentclass[11pt,a4paper]{article}
\usepackage{polski}
\usepackage[utf8]{inputenc}
\usepackage{mathtools}
\usepackage{color}
\usepackage{graphicx}
\usepackage{transparent}
\mathtoolsset{showonlyrefs}
\title{Generated by kSzereg}
\date{}
\begin{document}
	\documentclass[11pt,a4paper]{article}
\usepackage{polski}
\usepackage[utf8]{inputenc}
\usepackage{mathtools}
\usepackage{color}
\usepackage{graphicx}
\usepackage{transparent}
\mathtoolsset{showonlyrefs}
\title{Generated by kSzereg}
\date{}
\begin{document}
	\input{output/zad3.tex}
\end{document}

\end{document}

\end{document}


\section        {Wnioski}

%\dodatkowo{Programy}
%tu programy
%\begin{verbatim}
%#include <stdio.h>
%int main()
%{
%   printf("Hello world\n");
%}
%\end{verbatim}              
   
\begin{thebibliography}{2}

\bibitem{Poz1} Jakaś pozycja literatury
\bibitem{InnPoz} Jakaś pozycja literatury

\end{thebibliography}
\end{document}        
